\documentclass{article}
\usepackage{fullpage,fourier,booktabs,amsmath}
\usepackage[english]{babel}
\usepackage[autostyle, english=american]{csquotes}
\MakeOuterQuote{"}
\title{Lab 1: Estimating Pi with Coconuts}
\author{Isabella Phung}
\begin{document}\maketitle

\section{Introduction}

The irrational number pi is available in programming and C in a variety of libraries. Utlizing the digit in most applications requires less than 10 digits. There are a wide variety of ways to estimate pi, one of which is illustrated in the following program known as monte_carlo.
The monte_carlo program was written by the instructors of the CSE13S Winter 2023 class and estimates pi by drawing a unit square, inscribing a circle within it and dropping random points within the square. By finding the ratio between the number of points within the circle in comparison to the number of points outside of the circle, we should get approximately \frac{\pi}{4}.

\section{Diagrams}


\section{Conclusion}


\end{document}
