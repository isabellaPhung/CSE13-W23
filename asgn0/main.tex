\documentclass{article}
\usepackage[utf8]{inputenc}

\title{cse13s essay}
\author{Isabella Phung}
\date{January 2023}

\begin{document}

\maketitle

\section{Introduction}
two forms of censorship, of velina, one of silence, one of noise
knowledge inspires people to act/imitate. a dangerous combo.
ex, belusconi politician wanted to hide criticism so had press focus on meaningless true things.
don't completely avoid suspicion, instead just barely imply things such that the point cannot be challenged.
stating true statements without a conclusion/opinion can cause unease and confusion. people cannot point fingers if you do not have a distinct opinion.
This is noise.
adverts can't be objective and factual regarding their product. they have to be memorable. wheter by creating a brand ex: catchy name or tune. or by making the advert stand out rather than the product
internet generates a lot of noise. hard to parse for real information.
censorship by deliberately placing news stories in hard to access/not popular forms of media.
there are people who cannot live without the noise. are drawn to the noise of the internet. people on phones constantly updating. ex: restaurants with both music and tv and it's noisy. it's like a drug.
information is powerful with silence. word of mouth still remains the most powerful form of promotion.
we need to study silence, how it affects our communications, what silence means in theatre, in story telling, in politics, silence as a threat, silence as agreement, as denial, silence in music. What does suspense mean and how does it sit in our society.
Don't consider words. consider silence

Eco's point is true in my opinion. Modern day society is heavily reliant on the internet for information. There is constantly new media, new shows, new things to talk about. it seems like theres an event everyday and huge stories seem like old news after three days. There are many different factors that contribute to this phenomenon but Twitter has been an especially prominent example as of late with Musk's 300 billion dollar purchase of the company. For many, twitter is their main form of entertainment, communication, and news source. with the brevity of posts, it's difficult to make serious, concrete arguments or announcements citing specific sources. The format of replying, quote replying, private retweeting, all of these different ways of adapting someone else's words onto your own page, well, you're not required to respond, or respond well. There's a culture of unintelectualism that has brewed for years. Tiktok, tumblr, twitter, are all prime examples, where anyone can make a post that spins up a tale or a "fact" for views, and people will immediately agree without taking the time to search for other sources, to question. this encourages influencers to post and talk about what's popular or what the audience wahts to talk about rather than what the audience should know about. this has persisted in other forms of media beyond internet social media, but it's on a very large scale that is influencing a generation of children to believe in what's easy rather than the truth.
there are other deep social factors that also contribute (non-standardized education making it difficult to push for critical reading skills across America and the world, socio-economic factors that make spending the time to search through info and inform oneself to be difficult for people of middle to low class.) there's this pressure to be informed and caught up and to know everything. you haven't caught up with this show? another thing to watch. How come there isn't more attention on this? if you scroll past, you're a terrible person. That's not to say it's bad to be informed, but it's suffocating for the average person. There's also a dopamine rush that comes with using the internet, when seeing something funny, when communicating and connecting with others. Eco does describe noise as a form of censorship but he does not necessarily distinguish real important news as not noise either. What's considered crucial will depend on the audience of course. but in many ways real news can contribute to the noise. much in the same way Eco uses the example of planting a bomb in order to hides ones crimes in the newspaper, the oversatuation of news creates noise. It's hard to sift and parse through stories to find what really matters when it seems like everything matters. But it's unrealistic to expect one person to be constantly up to date with the politics and nuance of every country on earth; even the big players. Even if media takes the time to go as in depth as possible into a story, they can't spend the air time to give a history lesson on each country. Huge events are often interconnected in ways that news just can't extrapolate upon.
Eco's conclusion focuses on the concept of silence. He admittedly not so much as explores it as he does simply provoke the reader to investigate it themselves, but he brings up a flurry of diferent situations and prompts in regards to what silence means in our culture and society. The first thoughts that come to mind with silence is inaction. But in a world where noise is easy to produce and so readily abundant, silence speaks volumes. 

\end{document}
